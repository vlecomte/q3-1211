\chapter{Tâche 2: Synthèse de l'ammoniac}

Dans cette tâche, il nous est demandé d'étudier la dernière étape
du procédé, c'est-à-dire la synthèse de l'ammoniac à partir de
gaz de synthèse, formé d'azote, d'hydrogène et d'argon.
Cette réaction est la réaction E sur la figure~\ref{fig:flowsheet2}.

Nous allons d'abord étudier les taux de conversion que nous pouvons obtenir
de façon théorique
dans le réacteur et déterminer s'il est possible de faire réagir l'entièreté
du \ce{N2} et du \ce{H2} présents,
en faisant varier la température et la pression.
Cette option n'étant pas réalisable,
nous allons proposer une solution de recyclage des réactifs pour
améliorer le rendement.
Ces calculs théoriques sont réalisés avec \matlab{}.

Ensuite, nous allons utiliser le logiciel \aspen{} pour modéliser le procédé
que nous proposons de manière plus précise,
et comparer les résultats obtenus.
Nous allons expliquer comment les hypothèses simplificatrices
ont faussé nos calculs théoriques, et comment il serait encore possible
d'améliorer la modélisation en prenant d'autres facteurs en considération.

\section{Approche théorique}

Dans cette sections nous allons étudier le processus de réaction-séparation
avec \matlab{}, d'abord sans recyclage, puis avec recyclage.

\subsection{Réaction seule}

Commençons donc par estimer le rendement que nous pourrions obtenir
sur cette dernière réaction:
\begin{equation}
    \label{eq:chem-e}
    \ce{N2 + 3H2 -> 2NH3}
\end{equation}
Jusqu'à présent, dans la tâche 1, nous avions
considéré cette réaction comme complète.
Il faut donc maintenant résoudre les relations d'équilibre,
et étudier le résultat en fonction de la température et de la pression.

Pour cela, nous allons procéder comme pour les réactions du reformage primaire
en section~\ref{ssec:expr-eq}.
Soit $K\ind{E}$ la constante d'équilibre
de cette réaction. On a alors la relation:
\begin{equation}
    K\ind{E} = \frac{a_\ce{NH3}^2}{a_\ce{N2}\,a_\ce{H2}^3}
    = \frac{p_\ce{NH3}^2}{p_\ce{N2}\,p_\ce{H2}^3}
\end{equation}

Soient $\dot{n}_\ce{NH3}$, $\dot{n}_\ce{N2}$, $\dot{n}_\ce{H2}$
et $\dot{n}\ind{tot}$ les débits d'ammoniac, d'azote, d'hydrogène
et le débit total sortant du réacteur.
En exprimant les pressions partielles selon les fractions molaires
et la pression totale, on obtient:
\begin{equation*}
    K\ind{E} = \frac{(\dot{n}_\ce{NH3}\,/\,\dot{n}\ind{tot})^2
    \ \times (p\ind{tot}\,/\,p\ind{ref})^2}
    {(\dot{n}_\ce{N2}\,/\,\dot{n}\ind{tot})\ 
    (\dot{n}_\ce{H2}\,/\,\dot{n}\ind{tot})^3
    \ \times\ (p\ind{tot}\,/\,p\ind{ref})^4}
\end{equation*}
ou encore:
\begin{equation*}
    K\ind{E} \times (p\ind{tot}\,/\,p\ind{ref})^2
    = \frac{\dot{n}_\ce{NH3}^2\ \dot{n}\ind{tot}^2}
    {\dot{n}_\ce{N2}\ \dot{n}_\ce{H2}^3}
    = K_{\mathrm{E},\,p\indt{tot}}
\end{equation*}

Soient $a$, $3a$ et $a/78$ les débits de \ce{N2}, \ce{H2} et \ce{Ar} entrants,
et soit $\epsilon$ le degré d'avancement de la réaction
(comme déjà défini dans la section~\ref{ssec:inco-eq}).
Alors, l'équation s'écrit encore:
\begin{equation*}
    K_{\mathrm{E},\,p\indt{tot}}
    = \frac{(2\epsilon)^2\,(4.01a-2\epsilon)^2}
    {(a-\epsilon)\,(3(a-\epsilon))^3}
\end{equation*}

De plus, si nous nommons $y$ le rapport $\epsilon / a$ de conversion
des réactifs, l'équation devient:
\begin{equation}
    K_{\mathrm{E},\,p\indt{tot}}
    = \frac{(2y)^2\,(4.013-2y)^2}{(1-y)\,(3(1-y))^3}
    = \frac{4y^2\,(4.013-2y)^2}{27(1-y)^4}
\end{equation}
on peut facilement la résoudre numériquement comme une recherche de racines
d'un polynôme de degré 4.

Si on prend les température et pression indiquées dans l'énoncé,
c'est-à-dire $T=750\,\kelvin$ et $p\ind{tot}=270\,\bbar$, on obtient un taux de conversion:
\begin{equation*}
    y = 0.4006
\end{equation*}
ce qui est assez médiocre: cela signifie que seuls $40\%$ des réactifs
réagissent effectivement, et $60\%$ sont simplement jetés.

\subsubsection{Analyse paramétrique}

Nous allons maintenant étudier l'effet d'un changement
de température ou de pression sur le rendement de la réaction,
dans des intervalles réalistes.

\begin{figure}
    \centering
    \includegraphics[width=0.8\textwidth]{img/param-press}
    \caption{
        Augmenter la pression permet d'améliorer le rendement,
        mais c'est sans doute techniquement plus difficile et cher.
    }
    \label{fig:param-press}
\end{figure}

Commençons par faire varier la pression entre $200\,\bbar$ à $400\,\bbar$, pour une température constante de $750\,\kelvin$.
On peut constater sur la figure~\ref{fig:param-press} que le rendement s'améliore quand la pression augmente. En effet, quand on analyse l'équation chimique de la synthèse d'ammoniac \eqref{eq:chem-e}, il y a moins de moles de gaz dans les produits que dans les réactifs, et donc augmenter la pression aidera la réaction à se déplacer vers les produits.

Mais nous allons voir que la pression n'a pas une aussi grande influence que la température sur la réaction chimique, car le rendement varie seulement d'une dizaine de pourcents pour une pression variant du simple au double.

\begin{figure}
    \centering
    \includegraphics[width=0.8\textwidth]{img/param-temp}
    \caption{
        Baisser la température permet d'améliorer le rendement,
        mais la réaction va probablement devenir trop lente pour s'approcher
        de son équilibre.
    }
    \label{fig:param-temp}
\end{figure}

Faisons maintenant varier la température entre $500\,\kelvin$ et $1000\,\kelvin$, pour une pression de $270\,\bbar$,
On peut observer dans la figure~\ref{fig:param-temp} que le rendement diminue quand la température augmente. En effet, comme il s'agit ici d'une réaction exothermique, au plus la température est élevée, au plus la réaction est favorisée vers les réactifs, ce qui diminue notre rendement.

On peut également constater que la température aura une plus grosse influence sur la réaction que la pression. Effectivement, on peut voir sur l'échelle des ordonnées que le rendement varie de plus de $70\%$.

\subsubsection{Conclusions}

La conséquence de ces observations semble simple:
il faut augmenter la pression et baisser la température.
Toutefois, il faut prendre du recul et ne pas considérer uniquement
les équilibres chimiques.

En effet, augmenter la pression n'améliore pas sensiblement l'équilibre,
et cela ajouterait probablement des difficultés techniques, et par conséquent
des coûts supérieurs. Puisque il nous est difficile d'estimer ces coûts et leur
impact sur le prix de production, nous allons adopter dans nos calculs suivants
la pression standard qui nous est donnée de $270\,\bbar$.

Baisser la température améliore nettement l'équilibre,
et faire cela n'augmenterait a priori pas les coûts de production,
mais la cinétique chimique serait ici le facteur limitant:
en effet, en pratique, la réaction a besoin de plusieurs chambres
avec catalyseurs pour s'approcher de l'équilibre.
Diminuer la température ne peut qu'empirer le problème.
De nouveau, il est difficile d'en estimer précisément l'effet,
et les critères à optimiser nous sont inconnus.
Nous allons donc adopter la température donnée de $750\,\kelvin$.

\subsection{Recyclage des réactifs}

Dans la section précédente, nous avons vu que le rendement de la réaction
est assez mauvais, et que changer les conditions d'opération n'aide pas
vraiment à améliorer la situation.
Il nous faut donc une autre solution.

Cette solution consiste à séparer l'ammoniac produit des réactifs par
condensation, puis à récupérer les réactifs ainsi isoler pour réagir
à nouveau. Dans cette section, nous allons étudier l'influence de cette
mesure en considérant que la séparation est parfaite, c'est-à-dire que
l'intégralité de l'ammoniac est séparée, et qu'aucune quantité d'azote,
d'hydrogène ou d'argon n'est évacuée en même temps.

Remarquons directement que si tout est recyclé et réintroduit dans le
réacteur, l'argon ne sortira jamais du réacteur, et va s'accumuler
jusqu'à causer un accident ou du moins arrêter la production.
Il faut donc qu'une partie des réactifs isolés soit quand même
purgée, afin de garder la quantité d'argon à un niveau raisonnable.
Par conséquent, après l'\emph{unité flash} qui sépare l'ammoniac du reste,
il faudra un \emph{splitter} qui sépare le flux en deux,
avec une certaine part de purge.

\subsubsection{Choix des variables}

Étant donné que le calcul de l'équilibre avec recyclage est assez complexe,
commençons par présenter nos variables.
\begin{itemize}
    \item $a$ représente le flux 
\end{itemize}

\section{Modélisation avec \aspen}

\section{Comparaison des résultats}
