\chapter{Tâche 2: Synthèse de l'ammoniac}

Dans cette tâche, il nous est demandé d'étudier la dernière étape
du procédé, c'est-à-dire la synthèse de l'ammoniac à partir de
gaz de synthèse, formé d'azote, d'hydrogène et d'argon.
Cette réaction est la réaction E sur la figure~\ref{fig:flowsheet2}.

Nous allons d'abord étudier les taux de conversion que nous pouvons obtenir
de façon théorique
dans le réacteur et déterminer s'il est possible de faire réagir l'entièreté
du \ce{N2} et du \ce{H2} présents,
en faisant varier la température et la pression.
Cette option n'étant pas réalisable,
nous allons proposer une solution de recyclage des réactifs pour
améliorer le rendement.
Ces calculs théoriques sont réalisés avec \matlab{}.

Ensuite, nous allons utiliser le logiciel \aspen{} pour modéliser le procédé
que nous proposons de manière plus précise,
et comparer les résultats obtenus.
Nous allons expliquer comment les hypothèses simplificatrices
ont faussé nos calculs théoriques, et comment il serait encore possible
d'améliorer la modélisation en prenant d'autres facteurs en considération.

\section{Approche théorique}

Dans cette sections nous allons étudier le processus de réaction-séparation
avec \matlab{}, d'abord sans recyclage, puis avec recyclage.

\subsection{Réaction seule}

Commençons donc par estimer le rendement que nous pourrions obtenir
sur cette dernière réaction:
\begin{center}
    \ce{N2 + 3H2 -> 2NH3}
\end{center}
Jusqu'à présent, dans la tâche 1, nous avions
considéré cette réaction comme complète.
Il faut donc maintenant résoudre les relations d'équilibre,
et étudier le résultat en fonction de la température et de la pression.

Pour cela, nous allons procéder comme pour les réactions du reformage primaire
en section~\ref{ssec:expr-eq}.
Soit $K\ind{E}$ la constante d'équilibre
de cette réaction. On a alors la relation:
\begin{equation}
    K\ind{E} = \frac{a_\ce{NH3}^2}{a_\ce{N2}\,a_\ce{H2}^3}
    = \frac{p_\ce{NH3}^2}{p_\ce{N2}\,p_\ce{H2}^3}
\end{equation}

\subsection{Recyclage des réactifs}

\section{Modélisation avec \aspen}

\section{Comparaison des résultats}
