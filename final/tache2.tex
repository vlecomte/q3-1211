\chapter{Tâche 2: Synthèse de l'ammoniac}

Dans cette tâche, il nous est demandé d'étudier la dernière étape
du procédé, c'est-à-dire la synthèse de l'ammoniac à partir de
gaz de synthèse, formé d'azote, d'hydrogène et d'argon.

Nous allons d'abord étudier les taux de conversion que nous pouvons obtenir
de façon théorique
dans le réacteur et déterminer s'il est possible de faire réagir l'entièreté
du \ce{N2} et du \ce{H2} présents,
en faisant varier la température et la pression.
Cette option n'étant pas réalisable,
nous allons proposer une solution de recyclage des réactifs pour
améliorer le rendement.
Ces calculs théoriques sont réalisés avec \matlab{}.

Ensuite, nous allons utiliser le logiciel \aspen{} pour modéliser le procédé
que nous proposons de manière plus précise,
et comparer les résultats obtenus.
Nous allons expliquer comment les hypothèses simplificatrices
ont faussé nos calculs théoriques, et comment il serait encore possible
d'améliorer la modélisation en prenant d'autres facteurs en considération.

\section{Approche théorique}

\section{Modélisation avec \aspen}

\section{Comparaison des résultats}
