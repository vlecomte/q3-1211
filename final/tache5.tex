\chapter{Tâche~5: Dimensionnement d'une soupape de sécurité}

\section{Conditions initiales}
\quest{Quelle est la pression normale de stockage?}
La température normale de stockage est de $20\,\celsius$.
Nous prenons donc la pression de vapeur
correspondante sur le graphe, soit $8\,\barg$.
La pression normale de stockage est alors simplement:
\begin{equation*}
    P\ind{stockage} = 8\,\barg = 9\,\bbar
\end{equation*}

\quest{Quelle sera la pression de stockage en été (à 30\,°C)?}
En regardant sur le graphe de pression vapeur,
on a cette fois une pression de $11\,\barg$.
La pression de stockage vaut donc:
\begin{equation*}
    P\ind{stockage} = 11\,\barg = 12\,\bbar
\end{equation*}

\section{Conditions en décharge}
\quest{Quelle sera la pression maximale de tarage de la soupape de sécurité ?}
La pression de design est de $15\,\barg$,
et il n'y a pas de contre-pression.
Cela fait une différence de $16\,\bbar$ au niveau de la soupape.
Sachant que la pression de tarage ne peut pas être supérieure
à la pression de design, sans information sur les marges de sécurité,
la pression maximale de tarage que l'on peut fixer est donc:
\begin{equation*}
    P\ind{tarage} = 16\,\bbar
\end{equation*}

\quest{Quelle sera la pression durant la décharge ?}
Comme la contre-pression est nulle,
la pression durant la décharge sera égale à la pression de tarage:

\begin{equation*}
    P\ind{décharge} = P\ind{tarage} = 16\,\bbar = 15\,\barg
\end{equation*}

\quest{Quelle sera la température du liquide
durant la décharge via la soupape ?}
En regardant sur le graphe de pression vapeur, on voit que la température
correspondant à $15\,\barg$ est de:
\begin{equation*}
    T\ind{décharge} = 40\,\celsius
\end{equation*}

\section{Surface de section de la soupape}
\quest{Quelle est la taille de la soupape nécessaire?}
Étant donné que seule de la vapeur d'ammoniac
va quitter le tank à travers la soupape,
nous considérons l'équation pour une phase gazeuse pure:
\begin{equation}
    \label{eq:big-surface}
    A=\frac{W}{CK_dP_1K_bK_c} \sqrt{\frac{TZ}{M}}
\end{equation}

\quest{Surface exposée aux flammes}
Pour calculer la surface de contact,
nous devons considérer la surface du tank qui est à la fois en contact
avec de l'ammoniac liquide, et moins de $7.62\,\meter$ au-dessus des flammes.
La niveau d'ammoniac étant de $8\,\meter$,
la hauteur des flammes est le facteur limitant.
Il suffit donc de calculer la surface du tank en dessous de $7.62\,\meter$,
qui est composée d'une partie sphérique et d'une partie cylindrique:
\begin{equation*}
    A\ind{ws} = A\ind{hémisphère} + A\ind{cylindre}
    = 4\pi r^2/2 + 2\pi r (h-r)
    = 144\,\meter\squared
\end{equation*}

\quest{Chaleur absorbée}
La chaleur absorbée est définie par l'équation suivante:
\begin{equation}
    Q = C_1 F A\ind{ws}^{0.82}
\end{equation}
où $C_1$ est une constante définie à $43200\,\watt\per\meter\squared$,
et $F$ est un facteur
d'environnement. Dans notre cas, le tank n'étant pas isolé,
ce facteur vaut $1$. Nous avons donc:
\begin{equation*}
    Q = 43200\,\watt\per\meter\squared \times 1 \times 144\,\meter\squared
    = 2.54\,\mega\watt
\end{equation*}

\quest{Débit sortant}
On connait le débit de gaz sortant $W$ grâce à la formule:
\begin{equation}
    W=\frac{Q}{\Delta H\ind{vap}}
\end{equation}
où $\Delta H\ind{vap}$ est l'enthalpie de vaporisation de l'ammoniac,
en $\joule\per\kilogram$.
En effet, il faut faire en sorte que la vapeur puisse sortir au rythme auquel
elle est créée par l'apport de chaleur.
On trouve une enthalpie de vaporisation de $1150\,\kilo\joule\per\kilogram$
en reportant la température sur le graphe.
Dès lors:
\begin{equation*}
    W = \frac{2.54\,\mega\watt}{1.15\,\mega\joule\per\kilogram}
    = 7.94\cdot10^3\,\kilogram\per\hour
\end{equation*}

\quest{Section de la soupape}
Il nous suffit maintenant d'appliquer la formule \eqref{eq:big-surface}.
Les différentes constantes sont déterminées comme ça.
\begin{itemize}
    \item $C$ est déterminée par l'équation
        \begin{equation}
            C = 0.03948\,\sqrt{k\left(\frac{2}{k+1}\right)^\frac{k+1}{k-1}}
        \end{equation}
        où $k$ est le rapport de chaleurs spécifiques $c_p/c_v$ de l'ammoniac;
        $C$ n'est pas adimensionnel mais les unités ne sont pas précisées;
    \item $K_d$ est le coefficient de décharge, défini à $0.975$;
    \item $P_1$ est la pression de décharge;
    \item $K_b$ vaut 1 pour les soupapes non équilibrées;
    \item $K_c$ vaut 1 pour les soupapes sans disque de rupture;
    \item $T$ est la température de décharge, en $\kelvin$;
    \item $Z$ est un facteur de compressibilité, défini dans l'énoncé à 1;
    \item $M$ est la masse molaire de l'ammoniac, en $\kilogram\per\kilo\mole$.
\end{itemize}
Cela donne une valeur:
\begin{equation*}
    \begin{aligned}
        A\ &=\ \frac{7.94\cdot10^3\,\kilogram\per\hour}
        {0.0266 \times 0.975 \times 16\bbar \times 1 \times 1}\ 
        \sqrt{\frac{(273.15+40)\,\kelvin \times 1}{17\,\kilogram\per\kilo\mole}}
        \\
        &=\ 680\,\milli\meter\squared = 1.05\,\inch\squared
    \end{aligned}
\end{equation*}

\quest{Choix de la soupape}
En regardant ensuite dans le tableau des soupapes standard,
on a sélectionné la taille de soupape J, qui a une aire
effective de $1.287\,\inch\squared$.

\quest{Si la pression de design de l'équipement était de 20barg, quel serait l’effet d’augmenter la pression de tarage de 5 bar et de la porter à 20 barg ?}
En refaisant les calculs pour une pression de $20\,\barg$, on obtient sur
les graphes une température de $50\,\celsius$
et une enthalpie de vaporisation de $1100\,\kilo\joule\per\kilogram$.
En refaisant les calculs, on obtient alors:
\begin{equation*}
    A = 550\,\milli\meter\squared = 0.853\,\inch\squared
\end{equation*}
On doit donc de nouveau prendre une soupape de type J.

\section{Influence de l'isolation}
\quest{Pour la première pression de tarage, quelle est l’influence d’isoler thermiquement le tank avec un isolant tel que le coefficient d’échange avec l’extérieur soit réduit à une valeur de $\mathbf{10\ W/m^2\,K}$ ?}
Cela changerait alors le facteur d'environnement $F$,
et donc la taille de la soupape.
Isoler thermiquement le tank sera évidemment bénéfique,
et cela est vérifié dans les tables:
avec une telle isolation, on obtient par proportionnalité $F=0.132$.%
\footnote{En effet, un coefficient d'échange de
$11.36\,\watt\per\meter\squared\,\kelvin$
donnait un coefficient $F=0.15$.}
Il faudra donc utiliser une soupape plus petite.
Comme l'aire de l'orifice est directement proportionnelle à ce facteur, on obtient
\begin{equation*}
    A = 0.132 \times 680\,\milli\meter\squared
    = 89.8\milli\meter\squared = 0.139\,\inch\squared
\end{equation*}
On peut donc sélectionner une soupape de type E,
qui a une aire effective de $0.196\,\inch\squared$.
