\chapter{Tâche~1 -- Rapport~1: Circuit de refroidissement}

\section{Bilan de matière}
\label{sec:matiere}

Il s'agit de produire de l'ammoniac (\ce{NH3}) à partir de dihydrogène (\ce{H2})
et de diazote (\ce{N2}).
Cela donne l'équation pondérée suivante:
\begin{center}
    \ce{N2 + 3H2 -> 2NH3}
\end{center}

On obtient facilement les masses molaires des différents composés à partir
des masses atomiques de l'hydrogène et de l'azote:
\begin{equation*}
    \begin{array}{lcr}
        M_{\ce{N2}} &=& 28\,\gram\per\mole\\
        M_{\ce{H2}} &=& 2\,\gram\per\mole\\
        M_{\ce{NH3}} &=& 17\,\gram\per\mole\\
    \end{array}
\end{equation*}

En un jour, l'unité produit $1000\,\ton$ d'ammoniac.
Cela correspond à
\begin{equation}
    \label{eq:mole-nh3}
    \frac{1000\,\ton}{17\,\gram\per\mole} =
    5.88\cdot10^7\,\mole\mbox{ de \ce{NH3}}
\end{equation}

En utilisant les coefficients stœchiométriques, on trouve que les quantités de
réactifs nécessaires sont:
\begin{align*}
    1/2 \times 5.88\cdot10^7\,\mole &=
    2.94\cdot10^7\,\mole \mbox{ de \ce{N2}} \\
    3/2 \times 5.88\cdot10^7\,\mole &=
    8.82\cdot10^7\,\mole \mbox{ de \ce{H2}} \\
\end{align*}

Ce qui correspond aux masses suivantes:
\begin{equation}
    \begin{aligned}
        2.94\cdot10^7\,\mole \times M_{\ce{N2}} &= 824\,\ton \mbox{ de \ce{N2}}
        \\
        8.82\cdot10^7\,\mole \times M_{\ce{H2}} &= 176\,\ton \mbox{ de \ce{H2}}
    \end{aligned}
\end{equation}

Les flux de réactifs entrants seront donc $824\,\ton\per$j de diazote
et $176\,\ton\per$j de dihydrogène.



\section{Bilan thermique}

Pour que le réacteur soit maintenu à une température stable,
il faut que la puissance $P\ind{produite}$ dégagée par la réaction
soit compensée par la puissance $P\ind{dissipée}$ évacuée par l'eau.


\subsection{Puissance produite}

La puissance dégagée par la réaction est proportionnelle
à la différence d'enthalpie molaire $\Delta H\ind{m,réaction}$,
et au débit de matière $n\ind{t,\ce{NH3}}$ d'ammoniac produit.
Plus précisément:
\begin{equation}
    P\ind{produite} = \Delta H\ind{m,réaction} \times n\ind{t,\ce{NH3}}
\end{equation}

Pour calculer $\Delta H\ind{m,réaction}$
il faut rechercher l'enthalpie standard de
formation du \ce{NH3} dans les tables (à $25\,\degreecelsius$)
puis l'adapter à une
température de $500\,\degreecelsius$ en utilisant les capacités thermiques des
réactifs et des produits.
Pour simplifier les calculs nous supposerons qu'ils ne dépendent pas de la
température.

En tenant compte des coefficients stœchiométriques, on obtient:
\begin{equation}
    \begin{aligned}
        \Delta H\ind{m,réaction} &= \Delta\ind{f}H\ind{m,\ce{NH3(g)}}^\circ
        + \left( C_{p,\mathrm{m}}^\ce{NH3(g)}
        - \frac{1}{2} \times C_{P,\mathrm{m}}^\ce{N2(g)}
        - \frac{3}{2} \times C_{P,\mathrm{m}}^\ce{H2(g)}\right)
        \Delta T \\
        &= -46.11\,\kilo\joule\per\mole
        + (-22.73\,\joule\per\kelvin\usk\mole) \times 475\,\kelvin \\
        &= -56.91\,\kilo\joule\per\mole
    \end{aligned}
\end{equation}
en utilisant $C_{p,\mathrm{m}}^\ce{NH3(g)}=35.06\,\joule\per\kelvin\usk\mole$,
$C_{P,\mathrm{m}}^\ce{N2(g)}=29.12\,\joule\per\kelvin\usk\mole$
et $C_{P,\mathrm{m}}^\ce{H2(g)}=28.82\,\joule\per\kelvin\usk\mole$.
\cite{atkins}

Quant à $n\ind{t,\ce{NH3}}$ il s'agit simplement du
nombre de moles de \ce{NH3} produites par jour,
donc en reprenant le résultat du calcul~\eqref{eq:mole-nh3}, il vaut:
\begin{equation*}
    n\ind{t,\ce{NH3}} = 5.88\cdot10^7\,\mole\per\mbox{j}
\end{equation*}


\subsection{Puissance dissipée}

De manière analogue, la puissance évacuée par la circulation d'eau est
le produit de la différence d'enthalpie molaire et du débit de matière de l'eau:
\begin{equation}
    P\ind{dissipée} = \Delta H\ind{m,eau} \times n\ind{t,eau}
\end{equation}

Si nous supposons la capacité thermique de l'eau constante, la différence
d'enthalpie est simplement cette capacité multipliée par la
différence de température:
\begin{equation}
    \Delta H\ind{m,eau} = C_{P,\mathrm{m}}^\ce{H2O(l)} \times \Delta T\ind{eau}
\end{equation}
avec $C_{P,\mathrm{m}}^\ce{H2O(l)} = 75.29\,\joule\per\kelvin\usk\mole$
\cite{atkins} et
$\Delta T\ind{eau} = 90\,\degreecelsius-25\,\degreecelsius = 65\,\kelvin$

Et le débit de matière est le quotient du débit en volume par le volume molaire
de l'eau:
\begin{equation}
    n\ind{t,eau} = Q\ind{eau} \ /\  V\ind{m}^\ce{H2O(l)}
\end{equation}
avec $V\ind{m}^\ce{H2O(l)} = 1\,\liter\per\kilo\gram \times 18\,\gram\per\mole
= 1.8\cdot10^{-5}\,\meter\cubed\per\mole$ et $Q\ind{eau}$ à déterminer.


\subsection{Résolution pour le débit}

Nous avons travaillé avec des puissances absorbées positives et des puissances
dégagées négatives. Dès lors, l'équilibre thermique s'écrit:
\begin{equation}
    P\ind{produite} + P\ind{dissipée} = 0
\end{equation}

Résolvons maintenant cette équation pour trouver le débit $Q\ind{eau}$:
\begin{align*}
    -P\ind{produite} = P\ind{dissipée}
    \quad &\Rightarrow \quad
    -\Delta H\ind{m,réaction} \times n\ind{t,\ce{NH3}} =
    \Delta H\ind{m,eau} \times Q\ind{eau} \ /\  V\ind{m}^\ce{H2O(l)} \\
    &\Rightarrow \quad Q\ind{eau} =
    -\frac{\Delta H\ind{m,réaction}}{\Delta H\ind{m,eau}}
    \times n\ind{t,\ce{NH3}} \times V\ind{m}^\ce{H2O(l)}
\end{align*}

En introduisant les valeurs chiffrées, cela donne:
\begin{equation}
    \begin{aligned}
        Q\ind{eau} &=
        -\frac{-56.91\,\kilo\joule\per\mole}{75.29\,\joule\per\kelvin\usk\mole
        \times 65\,\kelvin} \times 5.88\cdot10^7\,\mole\per\mbox{j}
        \times 1.8\cdot10^{-5}\,\meter\cubed\per\mole \\
        &= 1.23\cdot10^4\,\meter\cubed\per\mbox{j} \\
        &= 142\,\liter\per\second
    \end{aligned}
\end{equation}

En conclusion, le refroidissement du réacteur nécessitera un débit d'eau
d'environ 142 litres par seconde.

\section{Flow-sheet simplifié}

Nous présentons en figure~\ref{fig:flow-sheet} un flow-sheet simplifié
du procédé de production d'ammoniac que nous allons étudier.
Nous l'avons réalisé à l'aide d'une rapide recherche documentaire,
voir sources \cite{process-patent, epa, contaminants}.

\begin{figure}
    \tikzstyle{reaction} = [
    rectangle, draw, thick, rounded corners, fill=black!10,
    text width=12em, text centered,
    minimum height=5em
]
\tikzstyle{inout} = [
    node distance=5em
]

\tikzstyle{thinflow} = [
    draw, -latex'
]
\tikzstyle{flow} = [
    thinflow, thick
]

\tikzstyle{flowing} = [
    text centered,
    font=\footnotesize,
    inner sep=.5em,
]

\begin{tikzpicture}[node distance=23em, auto]
    
    % Reaction boxes
    \node [reaction] (primary) {
        Reformage primaire \\[.3em]
        \footnotesize{
            \ce{CH4 + H2O <-> CO + 3H2} \\
            \ce{CO + H2 <-> CO2 + H2}
        }
    };
    \node [reaction, right of=primary] (secondary) {
        Reformage secondaire \\[.3em]
        \footnotesize{
            \ce{2CH4 + O2 <-> 2CO + 4H2} \\
            \ce{H2 + 1/2O2 -> H2O}
        }
    };
    \node [reaction, below of=secondary, node distance=9em] (shift) {
        Conversion du \ce{CO} \\[.3em]
        \footnotesize{
            \ce{CO + H2O <-> CO2 + H2}
        }
    };
    \node [reaction, left of=shift] (sepaco2) {
        Décarbonatation \\[.3em]
    };
    \node [reaction, below of=sepaco2, node distance=9em] (methanation) {
        Méthanation \\
        \footnotesize{
            \ce{CO + 3H2 -> CH4 + H2O} \\
            \ce{CO2 + 4H2 -> CH4 + 2H2O}
        }
    };
    \node [reaction, right of=methanation] (cryo) {
        Séparation cryogénique
    };
    \node [reaction, below of=cryo, node distance=9em] (condens) {
        Condensation de l'eau
    };
    \node [reaction, left of=condens] (synthesis) {
        Synthèse de l'ammoniac \\
        \footnotesize{
            \ce{N2 + 3H2 <-> 2NH3}
        }
    };
    
    % Empty in-out boxes
    \node [inout, above of=primary] (source1) {};
    \node [inout, above of=secondary] (source2) {};
    \node [inout, above of=sepaco2] (exitco2) {};
    \node [inout, above of=cryo] (exitcryo) {};
    \node [inout, below of=condens] (exitwater) {};
    \node [inout, below of=synthesis] (output) {};
    
    % In-out flows (thin arrows)
    \path [thinflow] (source1) --
    node [flowing, left, pos=0.45]{
        \ce{CH4}, \ce{H2O} 
    }
    (primary);
    \path [thinflow] (source2) --
    node [flowing, left]{
        \ce{CH4}, \ce{H2O}
    }
    node [flowing, right]{
        \ce{N2}, \ce{O2}, \ce{Ar}
    }
    (secondary);
    \path [thinflow] (sepaco2) --
    node [flowing, left]{
        \ce{CO2}
    }
    (exitco2);
    \path [thinflow] (cryo) --
    node [flowing, left]{
        \ce{CH4}, \ce{Ar}
    }
    node [flowing, right]{
        \ce{N2} (excès)
    }
    (exitcryo);
    \path [thinflow] (condens) --
    node [flowing, left]{\ce{H2O}}
    (exitwater);

    % Main pathway (thick arrows)
    \path [flow] (primary) --
    node [flowing, above]{
        \ce{CO}, \ce{CO2}, \ce{H2}
    }
    node [flowing, below]{
        \ce{CH4}, \ce{H2O} (traces)
    }
    (secondary); 
    \path [flow] (secondary) --
    node [flowing, left, pos=0.35]{
        \ce{CO}, \ce{CO2}, \ce{H2},
    }
    node [flowing, left, pos=0.65]{
        \ce{H2O}, \ce{N2}, \ce{Ar}
    }
    node [flowing, right]{
        \ce{CH4} (traces)
    }
    (shift);
    \path [flow] (shift) --
    node [flowing, above]{
        \ce{CO2}, \ce{H2}, \ce{H2O}, \ce{N2}, \ce{Ar}
    }
    node [flowing, below]{
        \ce{CH4}, \ce{CO} (traces)
    }
    (sepaco2);
    \path [flow] (sepaco2) --
    node [flowing, left, pos=0.35]{
        \ce{H2}, \ce{H2O},
    }
    node [flowing, left, pos=0.65]{
        \ce{N2}, \ce{Ar}
    }
    node [flowing, right, pos=0.35]{
        \ce{CH4}, \ce{CO},
    }
    node [flowing, right, pos=0.65]{
        \ce{CO2} (traces)
    }
    (methanation);
    \path [flow] (methanation) --
    node [flowing, above]{
        \ce{H2}, \ce{H2O}, \ce{N2}, \ce{Ar}
    }
    node [flowing, below]{
        \ce{CH4} (traces)
    }
    (cryo);
    \path [flow] (cryo) --
    node [flowing, left]{
        \ce{N2}, \ce{H2}
    }
    node [flowing, right]{
        \ce{H2O}
    }
    (condens);
    \path [flow] (condens) --
    node [flowing, above]{
        \ce{N2}, \ce{H2}
    }
    (synthesis);
    \path [flow] (synthesis) --
    node [flowing, left]{
        \ce{NH3}
    }
    (output);

\end{tikzpicture}

    \caption{Flow-sheet simplifié d'un procédé de production d'ammoniac.
        La mention «(traces)» indique des
        réactifs résiduels issus d'une réaction incomplète.
    }
    \label{fig:flow-sheet}
\end{figure}
