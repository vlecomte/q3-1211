\documentclass[a4paper, oneside]{scrartcl}

\usepackage{../latex/mystyle}
\addbibresource{tache3.bib}

\begin{document}

\titlehead{}
\subject{}
\title{Tâche 3}
\subtitle{ Etude d’impact environnemental sur l’ensemble du procédé }
\author{Groupe 1211}
\publishers{}
\date{\today}

\dedication{}

\maketitle

%%Ici commence le document
Pour implanter un nouveau site industriel de production d'ammoniac, il est indispensable de réaliser une étude sur l'impact environnemental de notre procédé, non seulement en termes de consommation énergétique mais également en rejet de \ce{CO2} et d'autres produits.

\section{Consommation énergétique}
Par consommation énergétique, nous entendons l'énergie consommée par notre procédé pour produire de l'ammoniac.
 

Les réactions qui nécessitent de l'énergie sont les réactions endothermiques. Or, les seules réactions endothermiques de notre procédé sont les réactions du reformage primaire : pour que celles-ci aient lieu, le réacteur du reformage primaire est relié à un four dans lequel a lieu une réaction exothermique de combustion de méthane qui produit la chaleur nécessaire.
 

Le méthane est un combustible fossile, et est donc une ressource non renouvelable. Cela fait qu'il a également un impact environnemental de par son extraction.

\section{Emissions de \texorpdfstring{\ce{CO2}}{CO \texttwoinferior}}
\subsection{Problématique \cite{prob1} \cite{changclim}}
Au fil des années, le réchauffement climatique est devenu une préoccupation majeure. Car, même si beaucoup de phénomènes naturels sont encore incompris et très complexes, il y a assez de preuves scientifiques pour nous forcer à étudier en profondeur la problématique. Or, nous savons que les émissions de \ce{CO2} constituent une des principales causes de ce réchauffement.
Or, à cause de la croissance permanente des activités industrielles de l'Homme, la concentration de \ce{CO2} n'a pas cessé d'augmenter au fil des années.

Ce climatique a de multiples conséquences pour notre environnement:
\begin{itemize}
    \item fonte des glaciers
    \item élévation du niveau de la mer
    \item perturbations jusqu’à destruction de certains écosystèmes
    \item extinction d’espèces
    \item baisse de ressources en eau potable
\end{itemize}
et plus particulièrement dans les pays du Sud:
\begin{itemize}
    \item ouragans
    \item cyclones
    \item inondations
    \item sécheresses
\end{itemize}
et bien d'autres...

 
Une conséquence moins connue de l’impact de l’émission de \ce{CO2} est l’acidification des milieux marins. En effet, les océans absorberaient un tiers des émissions humaines de \ce{CO2}. Cet apport massif de \ce{CO2} entraîne une diminution du pH des eaux, ce qui les rend plus acides. Cette acidification a un effet immédiat sur diverses espèces.

D’un point de vue plus pessimiste, le réchauffement climatique pourrait rendre l’absorption du \ce{CO2} par l’eau plus difficile, mais cela reste un point considéré comme très improbable.

Il nous paraît donc primordial d'analyser notre rejet en \ce{CO2} lors du procédé, afin de le minimiser.

\subsection{Rejet de notre procédé}
Grâce à notre outil de calcul réalisé auparavant, établissant les flux entrants et sortants de matière et d'énergie, nous sommes à-même de calculer notre rejet en \ce{CO2}.

Celui-ci a lieu à deux moments lors de notre procédé : premièrement, dans le four lors de la combustion du méthane, et deuxièmement, lors de la séparation du \ce{CO2} et de \ce{H2O}.

Chaque jour, pour la production de 1000 tonnes de \ce{NH3}, le processus rejette 1143 tonnes de \ce{CO2}, et le four en rejette quant à lui 153 tonnes de \ce{CO2}.

\section{Analyse du rejet d'autres produits secondaires}
Les autres produits rejetés sont de l'eau et de l'argon.
 

L'argon est rejeté tout à la fin du procédé, au niveau de la synthèse de l'ammoniac. Il n'a aucun impact sur l'environnement : aucun dommage écologique et aucune conséquence environnementale ne sont connus, en sachant que le procédé rejette 15 tonnes de \ce{Ar} par jour (pour une production journalière de 1000 tonnes de \ce{NH3}), ce qui est négligeable.
 

L'eau n'est bien entendu pas polluante, mais elle ressort du four extrêmement chaude. Il faut donc éviter son rejet direct dans la nature, qui aurait un impact sur l'écosystème local.

Notre procédé, pour une production quotidienne de 1000 tonnes de \ce{NH3} à 1000 \kelvin, rejette 121 tonnes d'eau et le four en rejette quant à lui 125. Ces chiffres sont une sous-estimation de notre rejet réel d'eau car on ne tient pas en compte le refroidissement.

\section{Risques d'un éventuel rejet d'ammoniac \cite{ammo1} \cite{ammo2} \cite{ammo3}}
Normalement, nous ne rejetons qu'une quantité infime de \ce{NH3}, au niveau de la purge. Il est assez difficile, avec les données disponibles, d'évaluer exactement la quantité rejetée. Mais il faut minimiser celle-ci au maximum car l'ammoniac a un énorme impact écologique.
 

En effet, l'ammoniac est un très gros polluant des eaux de surface, des nappes phréatiques et des sols. Il est responsable des pluies acides, de l'acidification des lacs, des cours d'eau, des sols forestiers, etc. Ainsi, de plus en plus de lacs, notamment en Suède, au Canada, en Norvège... deviennent trop acidifiés et parfois "biologiquement morts" : ils deviennent transparents, et si on s'extasie parfois devant une telle transparence, parfois associée à tort à de la "pureté", on n'imagine en fait pas que ces lacs sont en très mauvaise santé et ne contiennent aucune trace de vie aquatique ou végétale.
 

Plus généralement, l'ammoniac est responsable d'une perte de la biodiversité assez déplorable.

\section{Recommandations \cite{evaleco} \cite{lossprev}}
Voici les solutions que nous proposons pour réduire l'impact de l'activité de production sur l'environnement.
\subsection{Par rapport au \texorpdfstring{\ce{CO2}}{CO \texttwoinferior}}
Nous pouvons valoriser le \ce{CO2} en le réutilisant, ce qui évitera du même coup de rejeter celui-ci.
\begin{enumerate}
\item \emph{Stockage du \ce{CO2}  \cite{capt} \cite{capt2}} : Si le \ce{CO2} est stocké dans le sol, il ne sera donc pas dégagé dans l'atmosphère et n'aura donc aucun impact sur l'environnement. 
\item \emph{Production de micro-algues \cite{algues}} : Les micro-algues sont des plantes microscopiques qui ont besoin de \ce{CO2} pour se nourrir. 
\item \emph{Liquéfaction du \ce{CO2}} : On peut liquéfier le \ce{CO2} pour le réutiliser pour faire des boissons gazeuses par exemple. 
\item \emph{Production de formamides \cite{sol1} \cite{sol2}} : On peut également convertir le \ce{CO2} en formamides, qui sont des molécules servant à la production de colles, de peintures, de produits textiles, etc.
\item \emph{Conversion du \ce{CO2} en kérosène \cite{sol3} \cite{sol4}} : Bien que ce prodédé n'est encore seulement qu'à l'état de projet, il est extrêmement prometteur : dans un réacteur solaire à haute température, une lumière concentrée simulant la lumière solaire est utilisée pour convertir du \ce{CO2} et de l'\ce{H2 0} en un gaz synthétique, qui sera lui converti par après en kérosène, utilisant le procédé "Fischer-Tropsch".
\end{enumerate}
\subsection{Par rapport aux produits secondaires}
\begin{enumerate}
\item \emph{Récupération de l'eau chaude} Nous pouvons récupérer l'eau chaude et la chaleur dégagée et proposer une collaboration avec une entreprise locale, qui pourrait se servir de cette eau pour les radiateurs, pour des douches, etc. Par exemple, nous avons vu lors de la visite de l'usine de biométhanisation de Tenneville qu'ils collaboraient avec une entreprise locale en leur fournissant leur eau chaude pour nettoyer des bâches. Une telle collaboration nous paraît une excellente idée pour valoriser la chaleur produite durant notre procédé.
\end{enumerate}

\printbibliography[heading=bibintoc]

\end{document}
