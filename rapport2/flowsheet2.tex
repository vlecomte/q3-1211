\tikzstyle{reaction} = [
    rectangle, draw, thick, rounded corners, fill=black!10,
    text width=13em, text centered,
    minimum height=6em
]
\tikzstyle{inout} = [
    rectangle,
    node distance=6em,
    font=\footnotesize
]
\tikzstyle{ioside} = [
    inout,
    node distance=10em
]

\tikzstyle{thinflow} = [
    draw, -latex'
]
\tikzstyle{flow} = [
    thinflow, thick
]

\tikzstyle{flowing} = [
    text centered,
    font=\footnotesize,
    inner sep=.5em,
]

\begin{tikzpicture}[node distance=9em, auto]
    
    % Reaction boxes
    \node [reaction] (primary) {
        Reformage primaire \\[.3em]
        \footnotesize{
            (A) \ce{CH4 + H2O <-> CO + 3H2} \\
            (B) \ce{CO + H2 <-> CO2 + H2} \\
            (équilibre à la sortie)
        }
    };
    \node [reaction, below of=primary] (secondary) {
        Reformage secondaire \\[.3em]
        \footnotesize{
            (C) \ce{2CH4 + O2 -> 2CO + 4H2} \\
            (considérée complète)
        }
    };
    \node [reaction, below of=secondary] (shift) {
        Réacteurs Water-Gas-Shift \\[.3em]
        \footnotesize{
            (D) \ce{CO + H2O -> CO2 + H2} \\
            (considérée complète)
        }
    };
    \node [reaction, below of=shift] (sepa) {
        Séparation de \ce{CO2} et \ce{H2O} \\[.3em]
        \footnotesize{
            (considérée complète)
        }
    };
    \node [reaction, below of=sepa] (synthesis) {
        Synthèse de l'ammoniac et séparation \\
        \footnotesize{
            (E) \ce{N2 + 3H2 -> 2NH3} \\
            (considérée complète)
        }
    };
    \node [reaction, right of=primary, node distance=18em, fill=red!10] (oven)
    {
        Four \\[.3em]
        \footnotesize{
            \ce{CH4 + 2O2 -> CO2 + 2H2O} \\
            (complète)
        }
    };
    
    % Empty in-out boxes
    \node [inout, above of=primary] (in12) {\ce{CH4}, \ce{H2O}};
    \node [ioside, left of=secondary] (in3) {air};
    \node [ioside, right of=sepa, node distance=11em] (out12) {\ce{CO2}, \ce{H2O}};
    \node [ioside, right of=synthesis] (out3) {\ce{Ar}};
    \node [inout, below of=synthesis] (out4) {\ce{NH3}};
    \node [inout, above of=oven] (inoven) {\ce{CH4}, \ce{O2}};
    \node [inout, below of=oven] (outoven) {\ce{CO2}, \ce{H2O}};
    
    % In-out flows (thin arrows)
    \path [thinflow] (in12) -- (primary);
    \path [thinflow] (in3) -- (secondary);
    \path [thinflow] (sepa) -- (out12);
    \path [thinflow] (synthesis) -- (out3);
    \path [flow] (synthesis) -- (out4);
    \path [thinflow] (inoven) -- (oven);
    \path [thinflow] (oven) -- (outoven);

    % Energy flow
    \path [thinflow] (oven) --
    node [flowing, above] {chaleur}
    (primary);

    % Main flows
    \path [flow] (primary) --
    node [flowing, right]{
        \ce{CH4}, \ce{H2O}, \ce{CO}, \ce{CO2}, \ce{H2}
    }
    (secondary); 
    \path [flow] (secondary) --
    node [flowing, right]{
        \ce{H2O}, \ce{CO}, \ce{CO2}, \ce{H2}, \ce{N2}, \ce{Ar}
    }
    (shift);
    \path [flow] (shift) --
    node [flowing, right]{
        \ce{H2O}, \ce{CO2}, \ce{H2}, \ce{N2}, \ce{Ar}
    }
    (sepa);
    \path [flow] (sepa) --
    node [flowing, right]{
        \ce{H2}, \ce{N2}, \ce{Ar}
    }
    (synthesis);

\end{tikzpicture}
