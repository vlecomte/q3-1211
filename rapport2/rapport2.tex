\documentclass[a4paper,12pt]{article}

\usepackage{../latex/mystyle}
\addbibresource{ref.bib}

\begin{document}

\begin{center}
\begin{tabu} to \textwidth {lX[c]r}
    Groupe 1211 & \large{\textbf{Rapport~2}} & 14 octobre 2014 \\
    \hline
\end{tabu}
\end{center}

\section{Introduction}

Notre projet étant d’étudier la production d’ammoniac, en commençant par la synthèse de celui-ci à partir du reformage, il nous a été nécessaire de passer par l’étape de la gestion du plant. Ce rapport présente donc nos calculs et codes MATLAB du bilan de matière, en fonction de la température dans le réacteur et de la quantité d’ammoniac produite, du bilan d’énergie et finalement le calcul du nombre de tubes nécessaires à l’entrée des différents réactifs. 

\section{Flow-sheet rempli}
\section{Bilan de matière}
\section{Bilan énergétique}
\section{Nombre de tubes}
\section{Outil de gestion}

\printbibliography[heading=bibintoc]

\end{document}
