\documentclass[a4paper,12pt]{article}

\usepackage{../latex/mystyle}

\begin{document}

\begin{center}
\begin{tabu} to \textwidth {lX[c]r}
    Groupe 1211 & \large{\textbf{Rapport~1}} & 24 septembre 2014 \\
    \hline
\end{tabu}
\end{center}

\section{Bilan de matière}
\label{sec:matiere}

Il s'agit de produire de l'ammoniac (\ce{NH3}) à partir de dihydrogène (\ce{H2})
et de diazote (\ce{N2}).
Cela donne l'équation pondérée suivante:
\begin{center}
    \ce{N2 + 3H2 -> 2NH3}
\end{center}

On obtient facilement les masses molaires des différents composés à partir
des masses atomiques de l'hydrogène et de l'azote:
\begin{equation*}
    \begin{array}{lcr}
        M_{\ce{N2}} &=& 28\,\gram\per\mole\\
        M_{\ce{H2}} &=& 2\,\gram\per\mole\\
        M_{\ce{NH3}} &=& 17\,\gram\per\mole\\
    \end{array}
\end{equation*}

En un jour, l'unité produit $1000\,\ton$ d'ammoniac.
Cela correspond à
\begin{equation}
    \label{eq:mole-nh3}
    \frac{1000\,\ton}{17\,\gram\per\mole} =
    5.88\cdot10^7\,\mole\mbox{ de \ce{NH3}}
\end{equation}

En utilisant les coefficients stœchiométriques, on trouve que les quantités de
réactifs nécessaires sont:
\begin{align*}
    1/2 \times 5.88\cdot10^7\,\mole &=
    2.94\cdot10^7\,\mole \mbox{ de \ce{N2}} \\
    3/2 \times 5.88\cdot10^7\,\mole &=
    8.82\cdot10^7\,\mole \mbox{ de \ce{H2}} \\
\end{align*}

Ce qui correspond aux masses suivantes:
\begin{align*}
    2.94\cdot10^7\,\mole \times M_{\ce{N2}} &= 824\,\ton \mbox{ de \ce{N2}} \\
    8.82\cdot10^7\,\mole \times M_{\ce{H2}} &= 176\,\ton \mbox{ de \ce{H2}} \\
\end{align*}

Les flux de réactifs entrants seront donc $824\,\ton\per$j de diazote
et $176\,\ton\per$j de dihydrogène.



\section{Bilan thermique}

Pour que le réacteur soit maintenu à une température stable,
il faut que la puissance $P\ind{produite}$ dégagée par la réaction
soit compensée par la puissance $P\ind{dissipée}$ évacuée par l'eau.


\subsection{Puissance produite}

La puissance dégagée par la réaction est proportionnelle
à la différence d'enthalpie molaire $\Delta H\ind{m,réaction}$,
et au débit de matière $n\ind{t,\ce{NH3}}$ d'ammoniac produit.
Plus précisément:
\begin{equation*}
    P\ind{produite} = \Delta H\ind{m,réaction} \times n\ind{t,\ce{NH3}}
\end{equation*}

Pour calculer $\Delta H\ind{m,réaction}$
il faut rechercher l'enthalpie standard de
formation du \ce{NH3} dans les tables (à $25\,\degreecelsius$)
puis l'adapter à une
température de $500\,\degreecelsius$ en utilisant les capacités thermiques des
réactifs et des produits.
Pour simplifier les calculs nous supposerons qu'ils ne dépendent pas de la
température.

En tenant compte des coefficients stœchiométriques, on obtient:
\begin{align*}
    \Delta H\ind{m,réaction} &= \Delta\ind{f}H\ind{m,\ce{NH3(g)}}^\circ
    + \left( C_{p,\mathrm{m}}^\ce{NH3(g)}
    - \frac{1}{2} \times C_{P,\mathrm{m}}^\ce{N2(g)}
    - \frac{3}{2} \times C_{P,\mathrm{m}}^\ce{H2(g)}\right)
    \Delta T \\
    &= -46.11\,\kilo\joule\per\mole
    + (-22.73\,\joule\per\mole\usk\kelvin) \times 475\,\kelvin \\
    &= -56.91\,\kilo\joule\per\mole \\
\end{align*}

Quant à $n\ind{t,\ce{NH3}}$ il s'agit simplement du
nombre de moles de \ce{NH3} produites par jour,
donc en reprenant le résultat du calcul~\eqref{eq:mole-nh3}, il vaut:
\begin{equation*}
    n\ind{t,\ce{NH3}} = 5.88\cdot10^7\,\mole\per\mbox{j}
\end{equation*}


\subsection{Puissance dissipée}

De manière analogue, la puissance évacuée par la circulation d'eau est
le produit de la différence d'enthalpie molaire et du débit de matière de l'eau:
\begin{equation*}
    P\ind{dissipée} = \Delta H\ind{m,eau} \times n\ind{t,eau}
\end{equation*}

Puisqu'il s'agit d'un simple changement de température, la différence
d'enthalpie est simplement la capacité thermique de l'eau multipliée par la
différence de température:
\begin{equation*}
    \Delta H\ind{m,eau} = C_{P,\mathrm{m}}^\ce{H2O(l)} \times \Delta T\ind{eau}
\end{equation*}
avec $C_{P,\mathrm{m}}^\ce{H2O(l)} = 75.29\,\joule\per\mole\usk\kelvin$ et
$\Delta T\ind{eau} = 90\,\degreecelsius-25\,\degreecelsius = 65\,\kelvin$

Et le débit de matière est le quotient du débit en volume par le volume molaire
de l'eau:
\begin{equation*}
    n\ind{t,eau} = Q\ind{eau} \ /\  V\ind{m}^\ce{H2O(l)}
\end{equation*}
avec $V\ind{m}^\ce{H2O(l)} = 1,\liter\per\kilo\gram \times 18\,\gram\per\mole
= 1.8\cdot10^{-5}\,\meter\cubed\per\mole$ et $Q\ind{eau}$ à déterminer.


\subsection{Résolution pour le débit}

Nous avons travaillé avec des puissances absorbées positives et des puissances
dégagées négatives. Dès lors, l'équilibre thermique s'écrit:
\begin{equation*}
    P\ind{produite} + P\ind{dissipée} = 0
\end{equation*}

Résolvons maintenant cette équation pour trouver le débit $Q\ind{eau}$:
\begin{align*}
    -P\ind{produite} = P\ind{dissipée}
    \quad &\Rightarrow \quad
    -\Delta H\ind{m,réaction} \times n\ind{t,\ce{NH3}} =
    \Delta H\ind{m,eau} \times Q\ind{eau} \ /\  V\ind{m}^\ce{H2O(l)} \\
    &\Rightarrow \quad Q\ind{eau} =
    -\frac{\Delta H\ind{m,réaction}}{\Delta H\ind{m,eau}}
    \times n\ind{t,\ce{NH3}} \times V\ind{m}^\ce{H2O(l)} \\
\end{align*}

En introduisant les valeurs chiffrées, cela donne:
\begin{align*}
    Q\ind{eau} &=
    -\frac{-56.91\,\kilo\joule\per\mole}{75.29\,\joule\per\mole\usk\kelvin
    \times 65\,\kelvin} \times 5.88\cdot10^7\,\mole\per\mbox{j}
    \times 1.8\cdot10^{-5}\,\meter\cubed\per\mole \\
    &= 1.23\cdot10^4\,\meter\cubed\per\mbox{j} \\
    &= 142\,\liter\per\second \\
\end{align*}

En conclusion, le refroidissement du réacteur nécessitera un débit d'eau
d'environ 142 litres par seconde.

\end{document}
