\documentclass[a4paper,12pt]{article}

\usepackage[utf8]{inputenc}
\usepackage[T1]{fontenc}
\usepackage[frenchb]{babel}

\usepackage{tabu,enumerate,paralist,chemfig}
\usepackage{SIunits}
\usepackage[version=3]{mhchem}

\begin{document}

\begin{center}
\begin{tabu} to \textwidth {lX[c]r}
    Groupe 1211 & Rapport~1 & 24 septembre 2014 \\
    \hline
\end{tabu}
\end{center}

\section{Bilan de matière}

Il s'agit de produire de l'ammoniac (\ce{NH3}) à partir de dihydrogène (\ce{H2})
et de diazote (\ce{N2}).
Cela donne l'équation pondérée suivante:
\[\ce{N2 + 3H2 -> 2NH3}\]

On obtient facilement les masses molaires des différents composés à partir
des masses atomiques de l'hydrogène et de l'azote:
\[
    \begin{array}{lcr}
        M_{\ce{N2}} &=& 28\,\gram\per\mole\\
        M_{\ce{H2}} &=& 2\,\gram\per\mole\\
        M_{\ce{NH3}} &=& 17\,\gram\per\mole\\
    \end{array}
\]

En un jour, l'unité produit $1000\,\ton$ d'ammoniac.
Cela correspond à
\[\frac{1000\,\ton}{28\,\gram\per\mole} = 5.88\,10^7\,\mole\mbox{ de \ce{NH3}}\]

En utilisant les coefficients stœchiométriques, on trouve que les quantités de
réactifs nécessaires sont:
\[
    \begin{array}{lcr}
        \frac{1}{2} 5.88\,10^7\,\mole &=& 2.94\,10^7\,\mole
        \mbox{ de \ce{N2}} \\
        \frac{3}{2}
    \end{array}
\]

\end{document}
